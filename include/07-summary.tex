%Объем заключения не менее 3 стр.

\anonsection{Заключение}

В рамках данной работы был разработан и реализован алгоритм стегоанализа цветных изображений с использованием свёрточной нейронной сети. Кроме того, были реализованы известные нейросетевые алгоритмы стегоанализа изображений в оттенках серого GNCNN и нейронная сеть с двумя свёрточными слоями, впоследствии модифицированные для работы с цветными изображениями.

Исходя из результатов сравнения вышеупомянутых стегоаналитических алгоритмов, можно сделать предположение о том, что ввиду использования в GNCNN скользящих окон малого размера данная модель нейронной сети лучше подходит для анализа контейнеров, заполненных при помощи блочных алгоритмов, а нейронная сеть с двумя свёрточными слоями ввиду большого размера окна свёртки лучше подходит для стегоанализа алгоритмов, неравномерно распределяющих стегосигнал в пространственной области контейнера.

На примере комбинированной свёрточной сети показана практическая эффективность процедуры предварительной высокочастотной фильтрации. Направлением дальнейших исследований может быть поиск критерия оптимальности фильтра предварительной обработки.

Дополнительного исследования также требует факт более устойчивого детектирования алгоритмов, производящих встраивание в три канала цветного изображения, чем алгоритмов, производящих встраивание только в один канал, нейронной сетью GNCNN.

\clearpage
