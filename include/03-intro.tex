%Общий объем введения не менее 3 стр.

\anonsection{Введение}

На протяжении последнего десятилетия стегоанализ изображений является одним из активно развивающихся направлений в области цифровой стеганографии. Его главной задачей является обнаружение присутствия стегосообщений в цифровых изображениях. Существует множество разнообразных стегоаналитических подходов, но наибольшей эффективностью характеризуется группа методов, основанных на статистической обработке анализируемых данных. Их целью является построение математической модели заполненного контейнера, однако эта задача крайне осложнена разнообразием стеганографических методов и вносимых ними искажающих воздействий на контейнеры. В качестве такой модели, в основном, используется многомерное множество признаков, а решение задачи стегоанализа сводится к выполнению двух шагов: извлечению существенных признаков стегоконтейнеров, позволяющих судить о наличии стегосигнала, и бинарной классификации, сопоставляющей входные объекты с классами «пустой стегоконтейнер» и «заполненный стегоконтейнер». Успех стегоанализа в такой форме напрямую зависит от того, насколько точно извлечённые признаки отражают искажающий эффект стеговстраивания.

Активный рост вычислительных мощностей открывает новые возможности для создания адаптивных алгоритмов стеговстраивания, стремящихся обеспечить наилучшие характеристики стегосистемы применительно к конкретному стегоконтейнеру. Это влечёт за собой усложнение статистических зависимостей между отдельными элементами изображения, что, в свою очередь, существенно усложняет задачу создания математической модели заполненного стегоконтейнера и ручное конструирование признаков.

Математический аппарат глубоких искусственных нейронных сетей способен решить эту проблему путём автоматизации процесса извлечения признаков. Важным преимуществом таких признаков является их высокая релевантность, обеспечиваемая наличием в искусственной нейронной сети обратной связи между классификатором и экстрактором. К тому же, использование нейронных сетей значительно ускоряет процесс конструирования признаков и обеспечивает стегоаналитика набором неочевидных характерных особенностей заполненных стегоконтейнеров.

Актуальной является задача построения стегодетектора с использованием свёрточных нейронных сетей, способного работать с цветными изображениями, поэтому разработка соответствующего стегоаналитического алгоритма и стала целью данной работы.

Для достижения поставленной цели необходимо решить следующие задачи:
\begin{itemize}
\item проанализировать предметную область и изучить существующие стегоаналитические подходы;
\item разработать программную реализацию существующих алгоритмов стегоанализа, использующих свёрточные нейронные сети;
\item разработать собственный алгоритм стегоанализа цветных изображений на основе рассмотренных алгоритмов стегоанализа изображений в оттенках серого;
\item реализовать один или несколько методов стеговстраивания для тестирования алгоритма;
\item провести эксперимент и сравнить разработанный алгоритм с существующими.
\end{itemize}

\clearpage
