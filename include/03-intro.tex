%Общий объем введения не менее 3 стр.

\anonsection{Введение}

На протяжении последнего десятилетия стегоанализ изображений является одним из активно развивающихся направлений в области цифровой стеганографии. Его главной задачей является обнаружение присутствия стегосообщений в цифровых изображениях. Существует множество разнообразных стегоаналитических подходов, но наибольшей эффективностью характеризуется группа методов, основанных на статистической обработке анализируемых данных. Их целью является построение математической модели заполненного контейнера, однако эта задача крайне осложнена разнообразием стеганографических методов и вносимых ними искажающих воздействий на контейнеры. В качестве такой модели, в основном, используется множество многомерных признаков, а решение задачи стегоанализа сводится к выполнению двух шагов: извлечению существенных признаков стегоконтейнеров, позволяющих судить о наличии стегосигнала, и бинарной классификации, сопоставляющей входные объекты с классами «пустой стегоконтейнер» и «заполненный стегоконтейнер». Успех стегоанализа в такой форме напрямую зависит от того, насколько точно извлечённые признаки отражают искажающий эффект стеговстраивания.

Активный рост вычислительных мощностей открывает новые возможности для создания адаптивных алгоритмов стеговстраивания, стремящихся обеспечить наилучшие характеристики стегосистемы применительно к конкретному стегоконтейнеру. Это влечёт за собой усложнение статистических зависимостей между отдельными элементами изображения, что, в свою очередь, существенно усложняет задачу создания математической модели заполненного стегоконтейнера и ручное конструирование признаков.

Математический аппарат глубоких искусственных нейронных сетей способен решить эту проблему путём автоматизации процесса извлечения признаков. Важным преимуществом таких признаков является их высокая релевантность, обеспечиваемая наличием в искусственной нейронной сети обратной связи между классификатором и экстрактором. К тому же, использование нейронных сетей значительно ускоряет процесс конструирования признаков и обеспечивает обеспечивает стегоаналитика набором таких характерных особенностей заполненных стегоконтейнеров, которые сложно получить аналитически.

Данная работа посвящена программной реализации и исследованию алгоритмов стегоанализа изображений с использованием глубоких нейронных сетей, а также разработке собственного и сравнению его характеристик с известными. Отличительной особенностью разработанного алгоритма является высокая способность к различению пустых и заполненных цветных изображений-контейнеров.

\clearpage
