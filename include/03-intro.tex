%Общий объем введения не менее 3 стр.

\anonsection{Введение}

На протяжении последнего десятилетия стегоанализ изображений является одним из активно развивающихся направлений в области цифровой стеганографии. Его главной задачей является обнаружение присутствия стегосообщений в цифровых изображениях, зачастую подразумевающее два этапа: извлечение признаков и бинарную классификацию с классами «пустой стегоконтейнер» и «заполненный стегоконтейнер», причём успех стегоанализа напрямую зависит от того, насколько точно признаки отражают искажающий эффект стеговстраивания.

Активный рост вычислительных мощностей открывает новые возможности для создания адаптивных алгоритмов стеговстраивания, стремящихся обеспечить наилучшие характеристики стегосистемы применительно к конкретному стегоконтейнеру. Это влечёт за собой усложнение статистических зависимостей между отдельными элементами изображения, что, в свою очередь, усложняет задачу ручного конструирования признаков.

Математический аппарат глубоких искусственных нейронных сетей способен решить эту проблему путём автоматизации процесса извлечения признаков. Данная работа посвящена программной реализации и исследованию алгоритмов стегоанализа изображений с использованием глубоких нейронных сетей.

\clearpage
