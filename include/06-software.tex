%Общий объем раздела 15-25 стр. (x2, если разделы объединены)

\section{Описание разработанного программного обеспечения и экспериментальные исследования}

В рамках данной работы были реализованы три модели нейронных сетей для стегоанализа изображений в оттенках серого: GNCNN~\cite{GNCNN}, нейронная сеть с двумя свёрточными слоями~\cite{FrenchCNN} и собственная модель. Впервые были получены свёрточные ИНС для стегоанализа цветных изображений путём модификации вышеперечисленных моделей.

Также была произведена оценка различающей способности данных нейронных сетей.

\subsection{Структура разработанного программного обеспечения}

print(sys.version)
print(keras.__version__)
print(tf.__version__)
print(livelossplot.__version__)

3.5.2 (default, Nov 12 2018, 13:43:14) 
2.2.4
1.13.1
0.3.4


\subsection{Функциональные возможности разработанного программного обеспечения}
\subsection{Интерфейсная часть разработанного программного обеспечения}
\subsection{Перечень алгоритмов стеговстраивания по отношению к которым решалась задача}
\subsection{Описание реализованных алгоритмов стегоанализа}
\subsection{Результаты сравнительного исследования альтернативных вариантов алгоритмов стегоанализа}

Целью проведения эксперимента было определение способности вышеописанных нейронных сетей различать пустые и заполненные различными способами стегоконтейнеры. Для заполнения использовалось следующее программное обеспечение:

\begin{itemize}
\item программная реализация метода создания цифровых водяных знаков на основе гетероассоциативных сжимающих преобразований (ГСП)~\cite{SirotaHIC};
\item собственная реализация метода относительной замены коэффициентов дискретного косинусного преобразования (ДКП) Коха и Жао~\cite{ZhaoKoch, KochZhao};
\item симулятор стеговстраивания с использованием алгоритма WOW~\cite{WOW}.
\end{itemize}

Для проведения эксперимента использовалась база данных изображений PPG-LIRMM-COLOR~\cite{PPG-LIRMM-COLOR}, состоящая из 10 000 изображений в разрешении 512×512 пикселей. База данных подверглась конвертации в формат изображений в оттенках серого с помощью программы ppmtopgm~\cite{ppmtopgm}. Затем из каждого изображения были вырезаны два фрагмента размером 256×256 пикселей: один из них использовался для обучения нейронной сети сжатия в составе реализации метода создания цифровых водяных знаков на основе ГСП, второй – для осуществления встраивания обученной сетью. В дальнейшем последний фрагмент также использовался для стеговстраивания с применением собственной реализации метода Коха и Жао и симулятора встраивания с использованием алгоритма WOW.

Таким образом, для обучения нейронных сетей были получены три выборки из 10 тыс. пустых и 10 тыс. заполненных одним из перечисленных стегоалгоритмов контейнеров.

Также были сформированы аналогичные выборки из 1 тыс. пустых и 1 тыс. заполненных контейнеров для оценки влияния объёма выборки на точность классификации.

Обучающая подвыборка составила 90~\% полученной выборки, валидационная – 10~\% во всех случаях.

В рамках эксперимента было произведено сравнение реализаций методов стеговстраивания по средней среднеквадратической ошибке и среднему проценту восстановленных посредством стегоизвлечения данных~\ref{table:1}, а также обучение GNCNN и нейронной сети с двумя свёрточными слоями с целью оценки их способности к классификации стегоконтейнеров~\ref{table:2}.

Для каждого стегоалгоритма указан параметр, характеризующий мощность стеговоздействия. Для метода на основе гетероассоциативных сжимающих преобразований им является амплитуда встраивания $ A_m $, для алгоритма Коха и Жао – разность между коэффициентами ДКП $ p $, кодирующая различие между логическими нулём и единицей встраиваемой битовой последовательности, для алгоритма WOW – количество встроенных битов, делённое на количество пикселей в изображении, $ \alpha $.

\begin{table}[h!]
\centering
    \begin{tabular}{| l | l | l |}
    \hline
    Стегоалгоритм & MSE & Процент восстановленных данных \\ \hline
    На основе ГСП ($ A_m = 0,016 $) & 0,1334 & 97,23~\% \\ \hline
    Алгоритм Коха и Жао ($ p = 1 $) & 0,5775 & 99,78~\% \\ \hline
    WOW ($ \alpha = 0,4 $) & 0,0389 & – \\ \hline
%    \hline
    \end{tabular}
\caption{Характеристики реализаций методов встраивания}
\label{table:1}
\end{table}

Процент успешно извлечённых после стеговстраивания данных для алгоритма WOW не указан ввиду использования симулятора, не производящего сокрытия информации, а лишь изменяющего пиксели соответственно алгоритму встраивания.

\begin{table}[h!]
\centering
    \begin{tabular}{| l | p{2cm} | p{2cm} | p{2cm} | p{2cm} |}
    \hline
    Стегоалгоритм & GNCNN (2 K) & GNCNN (20 K) & 2\=/хслойная НС (2 тыс.) & 2\=/хслойная НС (20 тыс.) \\ \hline
    На основе ГСП ($ A_m = 0,016 $) & 94,1~\% & 96,8~\% & 48~\% & 51,7~\% \\ \hline
    Алгоритм Коха и Жао ($ p = 1 $) & 50,7~\% & 93,1~\% & 100~\% & 99,8~\% \\ \hline
    WOW ($ \alpha = 0,4 $) & 50~\% & 49,6~\% & 85,5~\% & 96,3~\% \\ \hline
    \end{tabular}
\caption{Точность классификации стегоконтейнеров валидационной подвыборки}
\label{table:2}
\end{table}

Из~\ref{table:2} видно, что обе нейронные сети имеют практически одинаковую способность к различению пустых стегоконтейнеров и стегоконтейнеров, заполненных с помощью реализации метода Коха и Жао, на выборке из 20 тыс. изображений.

Однако успешно детектировать заполненные с помощью реализации метода на основе ГСП стегоконтейнеры оказалась способна только GNCNN, в то время как детектирование заполненных с применением симулятора встраивания WOW контейнеров успешно производит только нейронная сеть с двумя свёрточными слоями.

\clearpage
