%Объем заключения не менее 3 стр.

\anonsection{Заключение}

В рамках данной работы были реализованы алгоритмы стегоанализа изображений с использованием свёрточных нейронных сетей и проведён эксперимент по установлению их способности к различению пустых и заполненных стегоконтейнеров.

В рамках работы был создана программа для электродинамической симуляции методом конечных разностей во временной области,
способная эффективно использовать ресурсы ГПУ. При помощи неё было произведено тестовое моделирование распространения гармонического 
сигнала, излучаемого тонким симметричным вибратором в замкнутом счётном объёме как с граничными условиями идеального отражения, так и с PML. 
Также была разработана референсная ЦПУ-реализация и произведено сравнение 
производительности расчётов с использованием ЦПУ и ГПУ.

В ходе работы было выяснено, что использование графических процессоров 
обеспечивает увеличение производительности до 17 раз, причём в программном коде
не был реализован выровненный доступ к памяти, что является 
перспективным направлением разработки. Также, в отличие от ЦПУ, 
производительность графических процессоров меньше зависит от размеров счётного объёма.

\clearpage
